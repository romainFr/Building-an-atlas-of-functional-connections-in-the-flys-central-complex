% Use this header.tex file for:
% 1. frontmatter/preamble LaTeX definitions
%  - Example: \usepackage{xspace}
% 2. global macros available in all document blocks
%  - Example: \def\example{This is an example macro.}
%
% You should ONLY add such definitions in this header.tex space,
%  and treat the main article content as the body/mainmatter of your document
% Preamble-only macros such as \documentclass and \usepackage are
% NOT allowed in the main document, and definitions will be local to the current block.
\newcommand{\beginsupplement}{%
        \setcounter{table}{0}
        \renewcommand{\thetable}{S\arabic{table}}%
        \setcounter{figure}{0}
        \renewcommand{\thefigure}{S\arabic{figure}}%
     }